\documentclass{beamer} 
%\documentclass[handout]{beamer} 

% Michael Maier, 2016.
% CC-0

\usepackage[utf8]{inputenc}
\usepackage[ngerman]{babel}

\title{Android Googlefrei} 
\author{Michael Maier \textless Michael.Maier@mailbox.org\textgreater} 
\date{26. April 2016} 

\usetheme{Antibes}

\hypersetup{colorlinks=true,urlcolor=blue,linkcolor=white}

%\usebackgroundtemplatei{
%\includegraphics[width=\paperwidth,
%height=0.8\paperheight]{mag_map.png}
%}

\begin{document}

%\maketitle

\begin{frame} 


\begin{figure}
  \centering
  \includegraphics[width=.5\textwidth]{mag_map.png}
\end{figure}

\begin{center}
\Huge{OpenStreetMap\\}
\end{center}

\begin{center}
\Large{\emph{Selbsterfasste Daten als Grundlage für Navigation}}
\end{center}

\end{frame}


\section{Einleitung}


\begin{frame}{Vorstellung}

  \begin{itemize}
    \item Michael Maier \textless \href{mailto:Michael.Maier@mailbox.org}{Michael.Maier@mailbox.org}\textgreater
    \item Student an der TU Graz (Telematik)
\vspace{0.3cm}
    \item Linux-User (Debian/grml) seit 2004
    \item Organisiere Grazer Linuxtage seit 2011 mit
    \item OpenStreetMap als Hobby seit Juli 2010
    \item Leite den Grazer OSM-Stammtisch seit Mai 2011
\vspace{0.3cm}
    \item Freiberuflich OSM-Aufträge und Consulting
    \item "`Regular"' bei der Cryptoparty Graz
  \end{itemize}
\end{frame}

War lange Zeit Smartphone-Verweigerer

Unfreie Systeme wie iOS kommen sowieso nicht in Frage

Angefangen mit OpenMoko ... Da läuft auch Debian
  War ein guter Mobilcomputer, als Telefon leider nicht praktikabel (Sprachqualität)

Laaange Pause, Android lange mit Sorge beobachtet

Dann kam das Fairphone
  Es kommt gerootet und ohne Google
  Und es ist auch noch ein wenig Fairer als andere
  Android 4.2.2

    Darauf basieren meine Erfahrungen - jedes Telefon ist anders!
  Wovon bin ich ausgegangen: Gerootetes Phone ohne Play-Store

Fairphone2 
  leider ohne root \& mit Google
  Android 5.1

Einige Monate paralell zu 6310i als Organizer verwendet, bis ich es geschafft hab alles so einzurichten, wie vom Nokia gewohnt


Ich sag es gleich - ich hab keinen Google-Account, und kann darum keinen exakten Vergleich zwischen den Apps machen

Flag: root ja/nein 


Play Store -> F-Droid
  Ja, man Muß installation von "Fremd"-Apps erlauben ((Screenshot))

  Aber auch wer glaubt, Apps aus dem Play-Store sind sicher, liegt falsch
  Betrüger treiben mittels Android-VMs die Downloadzahlen und Bewertungen in die Höhe

  100\,\% Schutz vor Viren gibts keinen - alles was ihr machen könnt ist: 
    F-Droid baut aus den Sourcen - Ihr könnt euch anschauen welche Entwickler dahinterstecken.
    

Next Step: XPrivacy
  Braucht zuerst das Xposed-Framework, dann ein paarmal rebooten, übers Framework XPrivacy installieren
  Damit könnt ihr Apps Rechte wieder wegnehmen/eure Privaten Daten verschleiern
  Ist aber nur der erste level, sie versprechen keinen 100\,\%-Schutz, an viele Daten kommen die Apps auch auf anderem Wege

Sonstige Settings:
  WLAN -> Advanced -> Network notification => DISABLE
  Backup & reset -> BACKUP & RESTORE -> Back up my data => DISABLE
  Google logs (nearly, unless user deactivates sync) all Wifi hotspots AND passwords worldwide...

  Screenshots: AndroSS ((Root))

  Maps -> OsmAnd ((Screenshot))
    Fahrrad: BikeCitizens

  Kommen wir zum Thema "Cloud"

  Kontakte :: CardDav
    PoliteDroid

   Kalender -> CalDav mit ownCloud 
     zusätzlich praktisch: ICSdroid
  
   Foto-Backup/Dateisync: ownCloud

   SMS/Call-log-Backup: SMS Backup+ 
    
 Sonstiges
   Ersatz für Proprietäre VoIP-Hangouts/Skype: Plumble

  Instant Messenger: LibreSignal statt Whatsapp

Sonstige Goodies aus F-Droid, ohne die ich nicht mehr leben will:
  Barcode Scanner
  Hacker's Keyboard
  GLT-Begleiter
  ClipStack
  OpenVPN
  Red Moon - f.lux fürs Handy
  Silent Night ((root)) - Nacht: Flugmodus
  Smarter WiFi-Manager


letzte Folie:
  Denkt daran, dass man Vorträge bewerten kann!

  Hands-On: Montag Cryptoparty


% Folien zu
% * kurze OSM-Vorstellung, Geschichtliches, Motivation
%  1. OSM-Vorstellung
  % was ist es
  % wer steckt dahinter?
% Geschichtliches
  % Gegründet ... steve
  % user-wachstum
% Motivation
  % gegründet, weil es keine freien Geodaten gab
  % Wunschtraum: eine DB weltweit

\begin{frame}{Was ist OpenStreetMap}

\begin{itemize}
  \item OpenStreetMap (OSM) ist eine freie Weltkarte nach dem Wiki-Prinzip "`Wikipedia der Karten"'
    \begin{itemize}
      \item \emph{Eigentlich eine Geo-Datenbank}
    \end{itemize}
\pause
  \item Entsteht aus der Arbeit von \textgreater 2,5\,M Hobbykartografen "`\emph{Mapper}"'

 \item Das komplette "`planet file"' ist ca. 74\,GB groß (xml.bz2) (Mittwoch):
  \begin{itemize}
    \item 3.297.315.809 Nodes
    \item 339.736.038 Ways
    \item 4.126.808 Relations
  \end{itemize}

 \begin{center}
 \includegraphics[width=5.5cm]{sotm.jpg}
 \end{center}

\end{itemize}

\end{frame}

\begin{frame}{Warum OpenStreetMap?}

\hspace{0.5cm}Es beginnt 2004 mit einer Geschichte: 
  \vspace{0.3cm}

Ein Student ärgert sich, dass es in UK keine freien Geodaten gibt. 
  \vspace{0.3cm}

\parbox{9.5cm}{Die Daten auf streetmap.co.uk wurden mit Steuergeldern erstellt, man kann die Rohdaten jedoch nicht frei verwenden.}
\hfill
\raisebox{\dimexpr-\height+\baselineskip}{\includegraphics[height=1cm]{traurig.png}}

  \vspace{0.6cm}
\pause

Warum muss man für etwas, was bereits von der Allgemeinheit mit Steuergeld bezahlt wurde, nocheinmal bezahlen?
  \vspace{0.3cm}

\parbox{9.1cm}{Und darf es selbst dann nicht frei Nutzen? \\Doppelbesteuerung ist zumindest bei uns verboten?}
\hfill 
\raisebox{\dimexpr-\height+\baselineskip}{\includegraphics[height=1cm]{grantig.png}}

\pause

 \parbox{7.5cm}{\vspace{0.4cm}\hspace{0.5cm}$\Longrightarrow$ \hspace{0.5cm}Er gründet OpenStreetMap!} 
\raisebox{\dimexpr-\height+\baselineskip}{\includegraphics[height=1.2cm]{laugh.png}}

\end{frame}


\begin{frame}{Wer steht hinter OpenStreetMap}

  \begin{itemize}
    \item OpenStreetMap Foundation (Server, Rechtliche Vertretung)
      \pause
    \item Mapper ($\sim$60.000 aktiv), meist ohne Geo-Hintergrund
    \begin{itemize}
      \item Jährliche Konferenz - "`State of the Map"', heuer: Brüssel
    \end{itemize}
      \pause
    \item Universitäten
    \begin{itemize}
      \item Bakk-, Master- und Doktorarbeiten mit OSM
      \item Server-Hosting
    \end{itemize}
      \pause
    \item Organisationen, die Daten sponsern
    \begin{itemize}
      \item Firmen wie Yahoo/Bing, die Luftbilder zur Verfügung stellen
      \item Regierungen mit besseren Open-Data-Gesetzen als Österreich
  % BSP TIGER, USA
  % Dänemark, Hausnummern
  % Frankreich,Tschechien: Kataster
    \end{itemize}
      \pause
    \item Firmen die mit OSM arbeiten, z.B.:
    \begin{itemize}
      \item Geofabrik (de)
      \item MapBox (us)
      \item BikeCitizens (Graz)
    \end{itemize}
  \end{itemize}



\end{frame}

%  
%{
% \usebackgroundtemplate{\includegraphics[height=10cm]{Osmdbstats2_users.png}}
%
%\begin{frame}{Geschichte von OpenStreetMap}
%  \vspace{0.6cm}
%\begin{itemize}
%  \item Start des Projekts im August 2004 durch \emph{Steve Coast}
%  \item Dezember 2006 - Yahoo erlaubt abzeichnen
%  \item Juli 2007 - Erste Konferenz, "`State Of The Map"'
%  \item August 2007 - 10.000 Registrierte Benutzer
%  \item März 2009 - 100.000 Registrierte Benutzer
%  \item November 2010 - Bing erlaubt abzeichnen
%  \item Juli 2011 - Erste "`State Of The Map Europe"' in Wien
%  \item Januar 2013 - 1.000.000 Registrierte Benutzer
%  \item Feb 2013 - Switchover to 64-bit 
%  \item Gestern -  Registrierte Benutzer
%\end{itemize}
%
%\end{frame}
%}
%


\section{Wie funktioniert OpenStreetMap?}
% * Technogie, Datenmodell, Lizenz

\begin{frame}{Woher kommen unsere Daten?}

\begin{itemize}
  \item Ursprünglich: GPS-Tracks
  \item Freiwillige tragen ihr Wissen bei: Jeder weiß viel über seine Umgebung:
	\begin{itemize}
	  \item Hausnummern, Straßennamen,
	  \item Restaurants, Bars, POIs, \dots
  \end{itemize}
  \pause
  \item Bei Mapping-Parties werden \\ gezielt Gebiete verbessert
\end{itemize}

  \vspace{0.4cm}
 99\% Handarbeit!

  \vspace*{-2.9cm}
 \hfill \includegraphics[width=4.2cm]{alps_mp.jpg}


  \pause
\begin{itemize}
  \item Hin und wieder Importe aus Open Government Data
  \begin{itemize}
    \item USA, TIGER Data (2008)
    \item Dänemark, Hausnummern (laufend synchronisiert)
    \item Wien, Baumkataster
  \end{itemize}
\end{itemize}

\end{frame}

\subsection{Technologie}

% 100% freie Software
% → jeder kann den Software-Stack verwenden http://openaviationmap.org/
% Quality Assurance
% * Es gibt automatische Q/A-Tools
% * kaum Streitfälle - wenn dann Mailinglist, Data Working Group
% * 
% tolle Bilder herzeigen!
% * irgendein Zoo
% * 3D -FIXME
% * Tolle Kartenstile:
%     * OSM-Fr?
%     * stamen watercolor
%     * pistemap
%     * bicycle map
%     * OpenSeaMap



\begin{frame}{Serverinfrastruktur}
Es gibt eine zentrale Datenbank (PostgreSQL/PostGIS) für Schreibzugriffe (in GB).\\
\pause
Diese wird weltweit gespiegelt für Lesezugriffe mit unterschiedlichen Methoden:

\begin{itemize}
  \item API-Lesezugriffe über mehrere Spiegel-Server lastverteilt
  \item Rendering-Server nutzen eine lokale, minütlich aktualisierte Datenbank
  \begin{itemize}
    \item Tileserver über GeoDNS weltweit verteilt (meist von Sponsoren)
  \end{itemize}
  \item Extrakte zum Download siehe \href{http://wiki.osm.org/Planet}{wiki.osm.org/Planet}
  \item Für räumliche SQL-Abfragen: Overpass API, zB alle italienischen Restaurants in Wien
\end{itemize}

\end{frame}


\subsection{Lizenz}

\begin{frame}{Lizenz}

  Die Daten stehen unter der \emph{Open Database Licence} - Entspricht etwa Creative Commons - Attribution - Sharealike für Daten.

 \begin{center}
 \includegraphics[width=1cm]{ODbL.png}
 \hspace{2cm}
 \includegraphics[width=1.5cm]{cc-by-sa.pdf}
 \end{center}

\pause

\vspace*{-0.3cm}

\begin{itemize}
  \item Jeder darf die Daten, auch kommerziell verwenden, jedoch:
  \begin{itemize}
    \item Attribution "`OpenStreetMap \& Contributors, ODbL"' angeben!
    \item Share-Alike: Wer die Daten verändert, muss sie unter derselben Lizenz veröffentlichen!
    \item Diese "`virale Lizenz"' stellt sicher, dass Verbesserungen nicht in den Silos von Konzernen verschwinden, sondern der Allgemeinheit weiter zur Verfügung stehen
  \end{itemize}

\end{itemize}


\pause
Die Web-Karten auf \href{http://osm.org}{openstreetmap.org} sind CC-BY-SA.
\begin{itemize}
  \item Beachte Tile Usage Policy!
\end{itemize}


\end{frame}

%
%\begin{frame}{Versionierung}
%
%Der komplette Datenbestand steht unter Versionskontrolle.
%\begin{itemize}
%  \item Auszüge können für beliebige Zeitpunkte erstellt werden
%  \item Spiegel-DB mit inkrementellen diffs minütlich aktualisierbar
%  \item DB sicher gegen Korrumption durch parallele Edits durch Verwendung von Changesets
%  \begin{itemize}
%    \item Pro Tag werden $\sim$16.500 Changesets submitted
%  \end{itemize}
%  \item Für jedes Objekt ist seine gesamte Historie abrufbar
%\end{itemize}
%
%% \vspace*{-3cm}
% \hfill \includegraphics[width=8cm]{history.png}
%
%
%\end{frame}

%\begin{frame}{Toolchain für Web-Karten}
%
%Wie funktioniert die Kartenanzeige im Browser?
%\pause
%
%\begin{itemize}
%  \item Javascript-Framework (zB Leaflet) lädt on Demand Kacheln (Tiles) vom Server
%  \item Die Tiles werden von  Apache mit \emph{mod-tile} ausgeliefert
%  \item mod-tile kontaktiert den Queue-Manager \emph{Tirex} für Renderjobs
%  \item Tirex rendert mittels \emph{Mapnik}
%  \begin{itemize}
%    \item Mapnik-Stile sind XML, das mittels CartoCSS generiert wird
%  \end{itemize}
%  \item Mapnik bekommt die Daten von einer PostGIS-DB
%  \item PostGIS-DB wird minütlich vom Hauptserver upgedated
%\end{itemize}
%
%Siehe Howto auf \url{http://switch2osm.org/}
%
%\end{frame}
%
%
%% QA -TODO
%
%\begin{frame}{Qualitätssicherung}
%Ähnlich Wikipedia, jeder darf alles ändern!
%  \begin{columns}[c]
%    \begin{column}[T]{.35\textwidth}
%      \vspace{1cm}
%      \includegraphics[width=4.5cm]{unconnected.jpg} \\
%      {\TINY CC-BY \url{http://www.bodenseepeter.de}}
%    \end{column}
%    \pause
%    \begin{column}[T]{.7\textwidth}
%      \begin{itemize}
%        \item Jedoch konfliktfreier als bei Wikipedia:
%        \begin{itemize}
%          \item Es gibt in OSM nur "`Ground Truth"'
%          \item Eintrittsschwelle ist höher (keine Anonymous edits)
%        \end{itemize}
%        \item Erfahrene Mapper kontrollieren ihr Gebiet mittels RSS-Feed
%        \pause
%        \item Eingebautes Social Network: Jeder Mapper kann persönlich kontaktiert werden
%        \begin{itemize}
%          \item Diskussion über die Mailingliste
%        \end{itemize}
%        \pause
%        \item Automatische Qualitätssicherungs-Tools
%        \begin{itemize}
%          \item \href{http://keepright.ipax.at/report\_map.php?zoom=14&lat=48.20808&lon=16.37221}{keepright.ipax.at}
%        \end{itemize}
%      \end{itemize}
%
%    \end{column}
%  \end{columns}
%
%\end{frame}
%


\section{OpenStreetMap Nutzen}

\subsection{Rohdaten}

\begin{frame}{OSM-Daten Downloaden}

	Download von Rohdaten im osm-xml Format:
\begin{itemize}
	\item kleinen Bereich: \href{http://osm.org}{osm.org}, Export
	\item Full Planet: \href{http://planet.osm.org}{planet.osm.org}
	\item Länderextrakte: \href{http://download.geofabrik.de}{geofabrik.de}
	\item SQL-Like API: Overpass, Webinterface: \href{http://overpass-turbo.eu}{overpass-turbo.eu}
\end{itemize}
\pause
Export in andere Formate: 
\begin{itemize}
	\item Bilder (PNG, JPG, SVG, PDF): \href{http://osm.org}{osm.org}, "`Share"'-Icon rechts
	\item Shapefiles: \href{http://download.geofabrik.de}{geofabrik.de} (Limitierte Spalten)
	\item GeoJSON: \href{http://overpass-turbo.eu}{overpass-turbo.eu}
\end{itemize}

\end{frame}

\subsection{Dienste}

\begin{frame}{Dienste}
	Was bietet OpenStreetMap:
\begin{itemize}
	\item Web-Karten zum Einbetten als HTML: \href{http://osm.org}{osm.org}, "`Share"'-Icon rechts
		\pause
	\item Links auf jedes einzelne OSM-Objekt; Marker
		\pause
	\item Geocoder: \href{http://nominatim.osm.org}{nominatim.osm.org}, Suche auf osm.org
		\pause
	\item Routing-Dienste für Auto, Fahrrad, Rollstuhl, \dots
\end{itemize}

\pause
Apps:
\begin{itemize}
   \item  Android ( \textgreater 100) \url{http://wiki.osm.org/Android}
   \item  iPhone ( \textgreater 70 )  \url{http://wiki.osm.org/Apple\_iOS}
   \item  Windows Phone ( 18 ) \url{http://wiki.osm.org/Windows\_Phone}
   \item  Blackberry ( 10 ) \url{http://wiki.osm.org/BlackBerry\_OS}
 \end{itemize}

\end{frame}

\section{Selbsterfasste Daten als Grundlage für Navigation}

\begin{frame}{Was wird erfasst?}

Praktisch Alles was einen Geobezug hat!

\pause

\begin{itemize}
  \item Straßen- und Wegenetz, Schiffahrtsrouten, Skipisten, \dots
  \item Flächen (Bewuchs, Landnutzung, Schutzzonen)
  \item POI-Eigenschaften wie Kontaktdaten, Öffnungszeiten, Rollstuhleignung, \dots
\end{itemize}

\pause

Tagging: 

Jedes Element kann beliebige Anzahl Eigenschaften haben.
Diese "`Tags"' genannten key=value Paare sind Freitext -- z.B.:
\begin{itemize}
  \item highway = footway 
  \item footway = sidewalk 
  \item surface = paved 
\end{itemize}

Dadurch ist man zu 100\% flexibel - Standards werden im Wiki festgelegt, siehe \href{http://wiki.openstreetmap.org/wiki/DE:How\_to\_map\_a}{wiki/DE:How\_to\_map\_a}

\end{frame}

\begin{frame}{ Straßen und Wege}

Highway = *

\begin{itemize}
  \item Wegtypen: motorway, primary \dots  residential, footway, cycleway, path
  \item Einbahnen und ihre Ausnahmen (Radfahrer etc)
\pause
  \item Fahrspuren: "`Lanes"'-Tagging-Modell:
  \begin{itemize}
    \item Abbiegespuren
    \item Busspuren, Radstreifen
    \item Zeitabhängige Beschränkungen "`Bus 6-9h"'
    \item Anwendbar auf jede Eigenschaft, zB Breite
  \end{itemize}
 \vspace*{-2.6cm}
\hfill\includegraphics[width=2cm]{lanes.png}

\pause
  \item Geschwindigkeitsbegrenzung: maxspeed = *
  \item Oberfläche: Befestigt, Pflaster, Schotter, ...
  \item Abbiegebeschränkungen: Mittels Relation from/to/via
  \item Beleuchtet, Tonnage, Durchfahrtshöhe, Brücke/Tunnel
\end{itemize}

\end{frame}

\begin{frame}{Baustellen}

Straßen im Bau werden natürlich auch erfasst, als eigener Wegtyp.

\begin{figure}
  \centering
  \includegraphics[width=7cm]{suedguertel.png}
\end{figure}

\pause

\begin{itemize}
  \item Sollen so zeitnah wie möglich eingetragen werden.
  \item Eventuelle Ausnahmen (Fußgänger, foot=yes) berücksichtigen.
  \item Kleine Baustellen mit construction=minor taggen, wenn nicht komplett gesperrt.
\end{itemize}

\end{frame}

\begin{frame}{Navigationssoftware}

Siehe \url{http://wiki.osm.org/Routing}

 \vspace*{0.3cm}
Serversoftware:

 \vspace*{-0.3cm}
\begin{itemize}
  \item OpenRouteService \includegraphics[width=1cm]{ors.png}
  \item Open Source Routing Machine (OSRM) \includegraphics[width=2cm]{osrm.png}
  \item GraphHopper \includegraphics[width=2cm]{graphopper.png}
  \item \dots
\end{itemize}

\pause

Unterschiedliche Profile:
\begin{itemize}
  \item Auto - LKW, Anhänger, ...
  \item Fahrrad - MTB, Citybike, Rennrad
  \item Fußgänger
  \item Rollstuhlfahrer - E-Rolli, Handrolli, Sportlich, \dots
\end{itemize}

\end{frame}

\begin{frame}{Praktisches Beispiel: Rollstuhlrouting}

Was ist wichtig für Rollstuhlfahrer:

\begin{itemize}
  \item Bordsteinkanten: Max 3\,cm!
  \item Breite (90\,cm)
  \item Oberfläche (\dots Pflastersteine)
  \item Querneigung (max 3\,\%)
  \item Steigung (max 2,5\,\%, max. 6\,\% auf 10\,m)
\end{itemize}

\includegraphics[width=7cm]{rolliroute.png}

\end{frame}


  \subsection{ OpenStreetMap Verbessern}

\begin{frame}{OpenStreetMap Verbessern}

  Eine große Auswahl an Editoren steht fürs Web, Desktop- und Mobilnutzung zur Verfügung

  \begin{itemize}
    \item Web:
    \begin{itemize}
	    \item Hauptseite - Edit: iD (JavaScript)
      \item oder auch einfach nur Fehler melden mit dem Note-feature auf \href{http://osm.org}{osm.org}!
	      \pause
    \end{itemize}
    \item Mobile (Auswahl): Alle siehe  \href{http://wiki.openstreetmap.org/wiki/Android\#OpenStreetMap\_editing\_features}{Android}, \href{http://wiki.openstreetmap.org/wiki/Apple\_iOS\#OpenStreetMap\_editing\_features}{iOS}:
    \begin{itemize}
      \item Vespucci: Ausgewachsener Editor
      \item osmaptuner: Existierende POIs ergänzen
      \item OsmTracker: GPS-Tracks, Audio, schnell POIs hinzufügen
    \end{itemize}
  \item Desktop
    \begin{itemize}
      \item \href{http://josm.openstreetmap.de}{JOSM}
      \item \href{http://merkaartor.be}{Merkaartor}
      \item ArcGIS (seit 10.1)
    \end{itemize}
  \end{itemize}

\end{frame}



\begin{frame}{Hilfe}
Fragen? 
\begin{itemize}
  \item Dokumentation: \href{http://wiki.openstreetmap.org}{wiki.openstreetmap.org}
  \begin{itemize} 
    \item Mitmachen? \href{http://learnosm.org/}{learnosm.org}
  \end{itemize}
  \item Immer noch etwas unklar? $\Rightarrow$ Mailingliste \href{http://lists.openstreetmap.org/listinfo/talk-at}{talk-at}
 \vspace*{0.4cm}
  \item Weltweite \href{http://usergroups.openstreetmap.de/}{Stammtische}
  \begin{itemize}
    \item 1/Monat Graz
    \item 1/Monat Wien
    \item 1/Monat Innsbruck
  \end{itemize}
 \vspace*{0.4cm}
  \item Grazer Linuxtage, 29.-30. April

\end{itemize}

 \vspace*{-2.8cm}
\hfill\includegraphics[width=5cm]{Salzburg_stammtisch.jpg}

\begin{itemize}
  \item Konferenz: \href{http://stateofthemap.org/}{State of the Map}, 23.-25. September, Brüssel
\end{itemize}
\end{frame}

\section{Ende}

\begin{frame}{Vielen Dank für die Aufmerksamkeit!}

  Folien zur RegioMove 7.4.2016, Leoben
\vspace{1cm}

Erstellt mittels \LaTeX Beamer, Quelltext: \href{https://github.com/species/vortrag-osm-regiomove16}{Github/species/vortrag-osm-regiomove16}.
\vspace{1cm}

\href{mailto:michael.maier@mailbox.org}{Michael Maier}

Twitter: \href{https://twitter.com/osmgraz}{@osmgraz}
\vspace{1cm}

Folien unter: \includegraphics[width=1cm]{cc-by-sa.pdf}. 

Alle Daten ODbL, OpenStreetMap Contributors.

\end{frame}



\end{document}
